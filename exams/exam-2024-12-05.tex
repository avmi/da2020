\documentclass[12pt,a4paper]{article}

\usepackage[T1]{fontenc}
\usepackage[charter]{mathdesign}
\usepackage{amsmath,amsthm,enumitem,titlesec,xcolor}
\usepackage{microtype}
\usepackage[a4paper,margin=25mm]{geometry}
\usepackage[unicode]{hyperref}

\hypersetup{
    hidelinks,
    pdftitle={Distributed Algorithms},
    pdfauthor={Jukka Suomela},
}

\pagestyle{empty}

\newcommand{\q}[2]{\paragraph{\mbox{Question #1: }#2.}}
\newcommand{\sep}{{\centering \raisebox{-3mm}[0mm][0mm]{$*\quad*\quad*$}\par}}
\newcommand{\hl}[1]{\textbf{\emph{#1}}}
\newcommand{\cemph}[1]{\textbf{\emph{\boldmath #1}}}

\DeclareMathOperator{\diam}{diam}

\setitemize{noitemsep,leftmargin=3ex}

\titleformat{\paragraph}[runin]{\normalfont\normalsize\bfseries}{\theparagraph}{1em}{}

\begin{document}

\noindent
\emph{CS-E4510 Distributed Algorithms / Jukka Suomela\\
exam, 5 December 2024}

\paragraph{Material.}

You can bring one A4 paper (two-sided), with any content you want. No other material or equipment is allowed in the exam.

\sep

\paragraph{Definitions.}

Let $G = (V,E)$ be a path graph (i.e., a connected acyclic graph where all nodes have degree at most $2$).
A node labeling $f\colon V \to \{1,2,3,\dotsc\}$ is called \emph{boring} if the following holds: if you look at any maximal segment that consists of consecutive nodes labeled with $x$, then there must be exactly $x$ nodes in the segment.

So for example in a path with $6$ edges and $7$ nodes, these are some examples of boring labelings:
\begin{align*}
    &(1,2,2,1,2,2,1),
    &(1,2,2,3,3,3,1),\\
    &(3,3,3,4,4,4,4),
    &(1,5,5,5,5,5,1),\\
    &(2,2,5,5,5,5,5),
    &(7,7,7,7,7,7,7).
\end{align*}
On the other hand, these are not boring (there is a segment with to many consecutive $2$s and a segment with too few consecutive $3$s):
\begin{align*}
    &(1,2,2,2,2,2,1),
    &(1,2,2,3,3,2,2).
\end{align*}

\sep

\paragraph{Questions.}

What can you say about the task of finding a boring labeling in path graphs? Try to answer at least the following questions:
\begin{enumerate}[noitemsep]
    \item Can it be solved in the \cemph{deterministic PN model} in \cemph{path graphs}? How fast?
    \item Can it be solved in the \cemph{deterministic LOCAL model} in \cemph{path graphs}? How fast?
\end{enumerate}
Initially, each node only knows its degree (in particular, it does not know the number of nodes). Try to design an algorithm that is as fast as possible, and try to show that no faster algorithm exists. If your algorithm is not a constant-time algorithm, try to at least show that no constant-time algorithm exists.

\sep

\paragraph{Instructions.}

Your mathematical proofs and algorithm descriptions can be short and informal. It is enough that a friendly, cooperative reader can understand your idea correctly and see why it makes sense. You are free to refer to algorithms and results that we discussed in the lectures, course material, and exercises; there is no need to repeat any of their details.

If you cannot answer any of the above questions, please try to say \cemph{anything} meaningful (positive or negative) about boring labelings in any of the models that we have studied in the course.

\end{document}
